\documentclass[../../main.tex]{subfiles}
\begin{document}

\label{sec:wordlist}
\begin{itemize}
    \item \textbf{Pan-tilt system}: A robot with a motor to control a frame for panning and a motor to control a frame for tilting, refer to figure \ref{fig:Pan_Tilt_Drawing}.
    \item \textbf{Incremental encoder}: An encoder which does not keep track of the absolute position, but requires an external reference point.
    \item \textbf{Index signal}: A reference signal which provides a fixed reference point for the position.
    \item \textbf{Frame}: A set of aluminium extrusions bolted together which a motor can rotate. On the pan-tilt system there is two frames, referred to as the tilt-frame and pan-frame. They are similar but not identical.
    \item \textbf{FPGA}: Field Programmable Gate Array, an integrated circuit of logic blocks that can be reprogrammed.
    \item \textbf{SPI}: Serial peripheral interface. A serial communications protocol.
    \item \textbf{PWM}: Pulse width modulation. A way of varying the voltage by switching between on and off-states.
    \item \textbf{H-bridge}: Hardware component which enables switching the polarity of a voltage.
    \item \textbf{Saturation}: The point at which the the system cannot perform better due to physical limitations, e.g. a motor reaching its maximum angular velocity.
    \item \textbf{Controller}: The control algorithm responsible for regulation of a given system.
    %\item \textbf{Lag}: The delay between experiencing a change in reference before seeing that same change in the output of a regulated system. 
    \item \textbf{Single controller}: A controller design utilising only a single control loop as opposed to cascade control.
    \item \textbf{Cascade control}: A control design utilising several nested control loops.
    \item \textbf{Root locus}: Root curves.
    \item \textbf{Basys-3-kit}: A development kit which contains an FPGA and peripherals like switches, LEDS, buttons, etc.
    \item \textbf{Model}: 
    
    \item \textbf{Tiva C-series LaunchPad}: A microcontroller development kit with the TM4C123GH6PM microcontroller.
    
    
    \item \textbf{UART}: Universal asynchronous receiver transmitter.
    \item \textbf{Robot (in this report)}: see Pan-tilt system.
    
    \item \textbf{PID-controller}: Proportional, Integral, Derivative controller.
    \item \textbf{Feedforward}: Feedforward uses measurement of the reference to eliminate steady state error.
    \item \textbf{Real-time}: The ability to guarantee that a task can initiate to a given time.
    \item \textbf{FreeRTOS}: An open-source real time operating system.
    \item \textbf{C}: Programming language. 
    \item \textbf{Modules}: Self-contained software unit communicating through multiple input and output.
    \item \textbf{VHDL}: Very high speed integrated circuit Hardware Descriptive Language.
    
    \item \textbf{TM4C123GH6PMI}: The microcontroller used for the project.
    \item \textbf{Basys-3 Artix 7}: The FPGA used for the project.
    \item \textbf{Back emf}: Electromotive voltage which opposes the change of current which induced it. 
    \item \textbf{Microblaze}: Soft microprocessor core, which is designed for Xilinx FPGAs.
\end{itemize}

\end{document}