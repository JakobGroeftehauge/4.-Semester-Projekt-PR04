\documentclass[../../main.tex]{subfiles}
\begin{document}

% Key parts of the Microcontroller:
% - Using freeRTOS 
    %Why it a real time operaing system is needed
    % - Timing of system/controllers
    % How much real time is it?
    
% - Implementing a controller    
    % What is needed to go from model to actual implementation(control design)
    % 

% - Send data to particular slave using SPI
    % How its used with the controllers
    % Speed

% - Setpoints, protocol
    % 

\subsection{Microcontroller} \label{subsec:SystemImplementationMicroController}

\subsubsection*{System overview}
\label{subsec:SystemImplementationOperatingSystem}

As described in the section \ref{subsec:SystemImplemtationFPGA} the microcontroller is responsible for getting data from specific slaves at the FPGA. Making it imperative that the microcontroller is in control of the overall timing of the system. As described in the requirements section \ref{sec:Requirements}, it is decided to accommodate cascaded controller consisting of two PID-controllers on each motor, yielding a total of 4 tasks needed to be scheduled regarding the motor control. Furthermore the microcontroller also needs to handle the user interface. 



To implement the PID-controller on the microcontroller, the discretized controller must be.

In order for the discrete controllers to work the interval 

The four controller tasks need to 



This results in four tasks that need to run with predetermined time intervals in real-time. Furthermore a User interface (UI) task for handling user input and a SPI task for sending and receiving data, will need CPU time. By assigning a lower priority to the UI and SPI tasks it is ensured that tasks with higher priority is prioritised when scheduling. This would also keep timing of the time critical controller tasks intact. 

For the PID-controllers to be able to receive data from the FPGA, SPI is used as mentioned in section \ref{subsec:SystemImplemtationFPGA}. When data is received from the slaves, it is put in buffers accessible only by specific controllers in order for the cascade control to work. To avoid critical section problems with several tasks trying to access the same buffers at the same time, mutex' is implemented. Output from each cascade must then go back to SPI for it to be delivered to the correct slaves. Figure (\ref{fig:OverviewTaskDiagramSimple}) shows a simple overview of the tasks and how they are coupled. A more detailed task diagram can be found in the digital appendix section \ref{sec:digital_appendix}.

% System clock frequency set to 80Mhz
\begin{figure}[H]
    \centering
    \includegraphics[width=\textwidth]{Sections/System_Implementation/Images/OverviewTaskDiagramSimple.pdf}
    \caption{Simplified task diagram showing the different tasks running on the microcontroller}
    \label{fig:OverviewTaskDiagramSimple}
\end{figure}






\subsubsection*{Discrete PID-controller}
To implement a PID-controller on the microcontroller, it must be discretized. As described in section \ref{sec:Analysis} the Tustin's method is used.
%Bodeplot til at finde cl båndbredde til at finde minimums samplings frequency
From equation \ref{eq:eq:discrete_control} it is apparent that the controller is dependant on the sampling time $T$, which on the microcontroller is determined by how often data is requested from the slaves. The minimum sampling time is controlled by how often data is sampled and updated on the FGPA. Velocity data is sampled on each encoder edge, with 1080 encoder ticks every revolution with a duration of \SI{0,88}{\second} \cite{} at maximum velocity, the minimum sampling time is:
\begin{equation}
    T_{min} = \frac{0.88\,\mathrm{s}}{1080} = 0.814\,\mathrm{ms}
\end{equation}
The maximum sampling time for a smooth response of the discrete controller, need to be 20 to 30 times that of the closed-loop bandwidth(LINK). From figure (ref) $T_{max}$ is found to be:
\begin{equation}
    T_{max} = \frac{1}{\omega_b\cdot 30} = \frac{1}{3 \cdot 30} \approx 11.1\,\mathrm{ms}
\end{equation}
In a cascade coupling the inner controller has to update faster than the outermost. To comply with this and making sure that $T_{min} > T > T_{max}$, the sampling time for the velocity controller is chosen to be $T_{vel} = 1\,\mathrm{ms}$ and the sampling time for the position controller is $T_{pos} = 5\,\mathrm{ms}$. \\


The gearing between the motor and the frame on the robot is 1:3, so the frame experiences one revolution for every three revolutions the motor experiences. The angle of the motor, $ \theta_{m} $, is described by equation \ref{eq:angle_of_motor}, where $FPGA_{pos}$ is the encoder value provided by the FPGA.
\begin{equation}\label{eq:angle_of_motor}
     \theta_{m} = \frac{\pi}{180} \cdot \mathrm{FPGA_{pos}}
\end{equation}


However a conversion is made in the microcontroller to convert the result to have the units \[ \left[ \dot{\theta}_m \right] = \SI{}{\radian \per \second } \] with respect to the motor. The motor encoder on the motor has 360 encoder ticks per revolution, which means that the relation between the angle of the motor measured in radians and the encoder ticks is, $\pi / 180$. As a result the conversion that is made microcontroller is given by equation \ref{eq:velocity_motor}.
\begin{equation}\label{eq:velocity_motor}
    \dot{\theta}_{m} = \frac{\frac{\pi}{180}}{ \dot{\theta}_{\mathrm{FPGA}} }\cdot 10^{5}
\end{equation}

\subsubsection*{PID-controller Tasks}
% How and which values need to be transported where Reference output controlling PWM
The PID-controller tasks on the microcontroller are responsible for calculating the control signal, which is used to control the motors. For each controller to be able to function they need three buffers. One buffer holding a reference value and another providing a feedback value. The third buffer is responsible for storing the controller output. In case of a velocity controller the reference buffer contains output from a position controller, while the output buffer is a queue passing on values to SPI. Each buffer has a mutex attached which needs to be waited for or signaled upon using. Figure \ref{fig:OverviewTaskDiagramSimple} constitutes an illustration of the implementation of two PID-controllers in cascade for a single motor. In the actual implementation there are two cascade controllers, one for each motor.

Following the controller task functionality described on figure  \ref{fig:PIDControllerFlowchart}, feedback to a controller is supplied whenever the controller sends a request to SPI. The request must contain a slave ID indicating which slave to receive data from. The controller task is then put in a blocked state, while waiting for data. Whenever SPI signals that the requested data is put in the feedback buffer associated with the requesting controller, the controller task is put back into running state. After that the error is calculated. Because of possible variance in feedback due to the imprecise separation of edges in the encoder, as mentioned in section \ref{subsec:SystemImplemtationFPGA}, a second order lowpass Finite Impulse Response (FIR) filter is applied to the error. Since the error ideally would consist of a DC frequency component, the filter was designed using a cutoff frequency at $0.5\,\mathrm{Hz}$. The filter transfer function used is:
\begin{equation}
    H(s) = \frac{1}{0.0249s^2 + 0.9502s + 0.0249}
\end{equation}
The proportional term is then calculated.
%If noise from the encoder is imagined to be a pure sine wave at a single frequency $\omega_a$, then by differentiating the noise described by equation \ref{eq:Sin_noise} it is seen that the amplitude, is multiplied with the frequency $\omega_a$ demonstrating that high frequency noise can ruin performance of the system. 
The derivative term amplifies noise, especially high frequent noise. Because of this to further dampen noise, a fifth order lowpass FIR filter is applied to the derivative term, with a cutoff frequency at $0.5\,\mathrm{Hz}$. The filter transfer function used is:
\begin{equation}
    H(s) = \frac{1}{0.0555s^5 + 0.1666s^4 + 0.2777s^3 + 0.2777s^2 + 0.166s + 0.055}
\end{equation}
The use of filters when implementing a controller was initially not assumed to be of high importance, when aiming to achieve good system response. Later this has been discovered to be a wrong assumption and that the choices regarding design of the filters have been poor. In section \ref{sec:Discussion} design of filters will be discussed. \\



% \begin{equation}\label{eq:Sin_noise}
%     \begin{split}
%         y_{noise}(t) &= A\cdot sin(\omega_a t+\phi_a)\\
%         \frac{dy_{noise}(t)}{dt}&=A\cdot\omega_a\cdot sin(\omega_a t+\phi_a+90\degree)
%     \end{split}
% \end{equation}
% where $A$ is the amplitude, $\omega_a$ is the frequency, $\phi_a$ is the phase and $t$ is the time.

To make sure the integral term does not wind up when an error is present for an extended period of time, anti-windup by conditional integration is implemented. As seen on figure \ref{fig:PIDControllerFlowchart} the controller performs a saturation check before updating the output buffer. Is the output in saturation a flag is set and the output is set equal to either the negative or positive saturation value, depending on sign of the error. 


\begin{figure}[H]
    \centering
    \includegraphics[width=\textwidth]{Sections/System_Implementation/Images/PIDControllerFlowchart.pdf}
    \caption{Flowchart describing the code execution in the PID-controller-task on the microcontroller.}
    \label{fig:PIDControllerFlowchart}
\end{figure}





\subsubsection*{SPI Task}
For the microcontroller to communicate and get data from the FGPA, SPI must be set up.
The microcontroller have build in hardware that enables the microcontroller to transmit and receive data using SPI. It is set up as a master and with a transmit and receive bit-rate of 13.3 MHz, when communicating with the slaves located on the FPGA. 
To address the different slaves on the FPGA four pins are used as slave select which are used in parallel with the hardware controlled SS.

As default the SPI task is in the BLOCKED state. Only when a request from one of the controllers enters the SPI-queue, as seen on figure \ref{fig:OverviewTaskDiagramSimple}, the task is given CPU time. Following figure \ref{fig:SPIFlowchart} the request taken from the queue contains a slave ID and data. When the appropriate slave is selected as determined by the slave ID, both data is transmitted and received. If the slave ID has an associated buffer e.g. a velocity feedback buffer, then the received data is put in that buffer.  

\begin{figure}[H]
    \centering
    \includegraphics[width=0.8\textwidth]{Sections/System_Implementation/Images/SPIFlowchart.pdf}
    \caption{Flowchart describing the code execution in the SPI task on the microcontroller.}
    \label{fig:SPIFlowchart}
\end{figure}

\subsubsection*{UI Task}
The UI task is responsible for handling user inputs including set-points for the position controllers. Communication is initiated when the user transmits a command through UART to the microcontroller via the computer. The UI task is given lowest priority, meaning it is only given CPU time whenever no other tasks are scheduled. To facilitate user inputs a protocol has been defined for which the different commands are described in table \ref{tab:UART_UI_PROTOCOL}. With these commands the user is able to reset both motors, home a motor or command a motor to go to a given set-point. The UI-task consists of two states: "Waiting for command" and "Waiting for queue". The state machine is illustrated in figure
\ref{fig:UITaskStateDiagram}. 

\begin{table}[H]
    \centering
    
    \begin{tabular}{l|c|l|p{0.1\textwidth}|c} %{p{0.12\linewidth}p{0.4\textwidth}p{0.38\textwidth}}
              & Byte & Description & ASCII chars & Value  \\
        \hline
        RESET & 0 & BEGIN RESET COMMAND & 1 & "0" \\
        \hline
        HOME MOTOR  & 0 & HOME MOTOR COMMAND & 1 & "1" \\
                    & 1 & Motor number & 1 & 0-1 \\
        \hline
        GO TO POSITION & 0 & GO TO POSITION COMMAND & 1 & "3" \\
            & 1 & Motor number & 1 & 0-1 \\
            & 2,3,4,5 & Encoder position to go to & 4 & 0-1080 \\
        \hline
        ABORT & 0 & ABORT COMMAND & 1 & "4" \\
        \hline
        MAX PWM TILT & 0 & MAX PWM TILT COMMAND & 1 & "5" \\
       & 1 & Direction of rotation of tilt arm & 1 & 0-1
    \end{tabular}    
    
    \caption{UART Protocol showing the different commands available for the user of the system. "Motor number" chooses either the tilt motor ("0") or the pan motor ("1"). Direction of rotation has similar mechanics where "0" chooses CCW rotation and "1" chooses CW rotation of the tilt arm.}
    \label{tab:UART_UI_PROTOCOL}
\end{table}

\begin{figure}[H]
    \centering
    \includegraphics[width=\textwidth]{Sections/System_Implementation/Images/UITaskStateDiagram.pdf}
    \caption{State diagram for the UI-task on the microcontroller.}
    \label{fig:UITaskStateDiagram}
\end{figure}




\subsection{Conclusion}

Logic has been implemented on the FPGA to evaluate the position and speed of the motors, to provide a PWM-signal to control the motors and to communicate over SPI with the microcontroller. On the microcontroller FreeRTOS, a real-time operating system, has been implemented for easy separation of algorithms into tasks, that can be scheduled by a scheduling algorithm. Overall the embedded system includes a task to handle the SPI-communication with the FPGA, a task to handle UART-commands sent by the user and tasks to implement the four PID-controllers.   


\end{document}