\documentclass[../../main.tex]{subfiles}
\begin{document}
\selectlanguage{English}
\begin{abstract}
%The purpose of this project is to examine different design and implementation processes of controlling a pan-tilt system.

The purpose of this project is to design and implement control of a pan-tilt system. The system will be controlled through user input in the form of commands sent from a PC. Apart from expertise in the discipline of control theory, the project explores the disciplines embedded programming and digital design. It is required that the control algorithms are implemented in an embedded system and the control signals are sent to the digital system, which acts as an interface to the robot. An SPI-protocol handles the communication between the two devices.
The embedded system comes in the form of a Tiva C-series LaunchPad development kit with a TM4C123GH6PM microcontroller for which programs are written in C and on which a real-time operating system is implemented. The digital system comes in the form of a Basys-3 development kit with an Artix-7 FPGA for which logical expressions are written in VHDL.
An UART protocol is developed, that specifies commands available for the user to control the pan-tilt system. The project examines the response of the pan-tilt system using the Ziegler-Nichols and pole placement tuning methods for the PID controller as well as comparing the response of a single PID controller and two PID controllers in cascade. In addition the difference between modelling and implementation is examined by comparing the responses of simulated and implemented PID controllers. 

(Noget med konklusionen)

\end{abstract}

\end{document}
