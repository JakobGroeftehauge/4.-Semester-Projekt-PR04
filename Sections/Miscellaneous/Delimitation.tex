\documentclass[../../main.tex]{subfiles}
\begin{document}
As described in section \ref{sec:Analysis}, there are a broad range of solutions regarding controlling the system. Some of the solutions might be equally valid, but offer different approaches to the project. This section concerns the choices made in the project and will found the basis for the requirement specifications.

\subsection*{Physical Environment of the Project}
As mentioned in section \ref{sec:ProblemDescription} some requirements from the project description has to be met. To accommodate the use of both an microcontroller and an FPGA. The controllers will be implemented on the \textit{Tiva C-series launchpad} and the motor interface will be implemented in the Artix-7 FPGA on the Basys-3 development kit. This is due to experience from lectures concurrent with the project. While it is recognised that other and better solutions might be available, it has not been of high priority to find these due to the scope of the project. 


% The problem description section \ref{sec:ProblemDescription} dictates some requirements for the project. Such as the use of system, the requirement of using a micro controller and an FPGA. The given system provides attached encoders, which is to be used. The microcontroller on the \textit{Tiva C-series launchpad} is used to implement the controllers and the Artix-7 FPGA on the Basys-3 development kit is used to interface the motors. This is due to experience from lectures concurrent with the project. While it is recognised that other and better solutions might be available, it has not been of high priority to find these due to the scope of the project.

\subsection*{Controller}
As mentioned in section \ref{sec:Analysis} different types of controller methods are available. The modern control methods using state feedback offers a rather complex solution, where the knowledge of the system is crucial to achieve the wanted behaviour of the controller. If the model of the system does not comply with the actual system, the controller will be subject to a lot of tuning and tweaking, and with a vast number of tuning parameters it can be difficult and rather unintuitive to tune the controller. A method with less and more intuitive tuning parameters is desired. The classic control methods such as the PID controller offers exactly that. Using different tuning methods an estimation of the three gains is achieved, and with well-defined definitions for the function of each gain, the tweaking process for the actual system is rather intuitive. Thus it is chosen to work towards implementing PID controllers.

Tuning the PID controllers it is desired to examine different solutions and comparing these. For that reason a single tuning method wont be settled. It is desired to test the pole-placement method and Nichols-Ziegler. Likewise it is wanted to examine the behavior of the system using different constellations of the controllers. Thus it is worked towards implementing a single controller  as well as controllers in cascade for the pan and tilt on the micro controller.

To make it easier to design the PID controller honoring the set requirements for the system, the controller is designed in continuous time, after which it is converted into discrete time for the implementation on the micro controller. 

Initially all three states of the system was to be measured, namely the position, the velocity and the current. Being able to deduct the position and the velocity from the encoder output, as described in section \ref{sec:Analysis}, measuring the current is not as easy. Being a very fast signal it is doubt-full whether the micro controller or even the FPGA manages to cope with the speed. Additionally a lot of mechanical disturbances from the motors interferes with the signal making it unreliable and due to shortage of time and prioritising elsewhere, measuring of the current was discontinued. Thus a position and velocity controller is to be implemented.

Examining the difference between the implementation of a controller on the model and the actual system is also a priority for the project. In conclusion the focus of the project is to examine different methods and approaches of controlling the system rather than offer an actual product with a purpose. Thus a set-point for the controllers transmitted via UART from the computer to the micro controller is deemed sufficient as user input. The UART protocol is chosen due to previous experience with the protocol making it an obvious choice. 
%PID controller
%single, cascade
%Current
%Test a lot of different tuning methods



\subsection*{Micro Controller}
It is chosen to use FreeRTOS v10.2.1 as operating system for the micro controller. This is primarily due to the fact that the it is based on RTOS which provides key features that are advantageous when runing multiple controllers. Particularly; preemptive scheduling supporting multitasking, prioritization of tasks and by being a real-time OS having little buffer delay. Beyond that FreeRTOS provides functionality for queues, mutex' and semaphores. It is important to notice however that CPU utilisation cannot be too high since it could result in starvation.


\subsection*{FPGA}
In the project description it is stated how the FPGA should interact with the micro controller and the robot.
SPI is to be used as communication protocol between the micro controller and the FPGA. Therefor it is necessary to implement modules capable of constituting such a protocol. Beside handling the communication the FPGA must be able to deduct the position and velocity using the encoders signals and producing a desired PWM signal to the H-brigde. In accordance with the datasheet of the H-Bridge, L6203 \ref{sec:digital_appendix}, it requires a PWM frequency typical around 30 KHz and up to 100 KHz to function, meaning that the PWM signal from the FPGA must be in that frequency range. 
It is chosen to divide the program implemented in the FPGA in to small modules that only contains the necessary code specific for that module. The intent of this is to make most of the modules reusable controlling the pan frame as well as the tilt frame.

%Krav omkring reset til index pos.
For regulation the pan tilt system to a desired position it is essential to have a reference position. To be able to obtain the reference position functionality is to be build into the FPGA, to ensure that the pan-tilt system will return to a defined zero position. 
%Køre SPI 
%Sende PWM (opløsning) med bestemt frekvens
%Beregne position og hastighed
%Program opbygget i moduler (nemt er organisere og implementere på pan og tilt)

\end{document}