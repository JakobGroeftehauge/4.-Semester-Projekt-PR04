\documentclass[../../main.tex]{subfiles}
\begin{document}
As described in section \ref{sec:Analysis}, there are a broad range of solutions regarding controlling the system. Some of the solutions might be equally valid, but offer different approaches to the project. This section concerns the choices made in the project and will found the basis for the requirement specifications.

\subsection*{Physical Environment of the Project}
As mentioned in section \ref{sec:ProblemDescription} some requirements from the project description has to be met. The controllers will be implemented on the \textit{TM4C123GH6PM} microcontroller as part of the Tiva C-series LaunchPad development kit and the motor interface will be implemented on the \textit{Artix-7} FPGA on the Basys-3 development kit. This is due to experience from lectures concurrent with the project. While it is recognised that other and better solutions might be available, it has not been of priority to find these due to the scope of the project. 


% The problem description section \ref{sec:ProblemDescription} dictates some requirements for the project. Such as the use of system, the requirement of using a microcontroller and an FPGA. The given system provides attached encoders, which is to be used. The microcontroller on the \textit{Tiva C-series launchpad} is used to implement the controllers and the Artix-7 FPGA on the Basys-3 development kit is used to interface the motors. This is due to experience from lectures concurrent with the project. While it is recognised that other and better solutions might be available, it has not been of high priority to find these due to the scope of the project.

\subsection*{Controller}
As mentioned in section \ref{sec:Analysis}, different types of controller methods are available. The modern control methods using state feedback offers a rather complex solution, where a precise model of the system is crucial to achieve the wanted behaviour of the controller. If the model of the system does not comply with the physical system, the controller will be subject to a lot of tuning and tweaking. With a vast number of tuning parameters the tuning can be difficult. A classic control method such as the PID-controller offers a simpler design with fewer parameters to tune. Thus it is chosen to work towards implementing PID-controllers. When tuning the controllers it is desired to examine different methods of tuning and comparing these. It is desired to test the pole-placement method and the Ziegler-Nichols step response method. Likewise it is wanted to examine the behaviour of the system using different constellations of the controllers. Thus it is worked towards implementing a single controller as well as controllers in cascade, for the pan and tilt motors, on the microcontroller. 
The controllers will be designed in continuous time, after which, they transformed into discrete time using Tustin's method, for implementation on the microcontroller. 

Initially all three states of the system was to be measured, namely the position, the velocity and the current. The position and velocity can be deduced from the encoders, as described in section \ref{sec:Analysis}. The current is a high frequency signal making it difficult to measure and control using available hardware. For this reason measuring the current was discontinued. Thus position and velocity controllers are to be implemented.

It is assumed when increasing the sampling rate, that performance of a discrete controller only improves. Therefore it has been decided to run single controllers, as well as inner loops of a cascaded design, at a rate of \SI{1000}{\hertz}. The outer controller of a cascaded design will run at \SI{200}{\hertz}. It has later been discovered that this rate may be excessive, which will be discussed in section \ref{sec:Discussion}.

A priority of the project will be to examine the difference between the simulation using the model and the implementation on the physical system. In conclusion the focus of the project is to examine different methods and approaches of controlling the system rather than work towards a specific application. A reference for the controllers transmitted via UART from the computer to the microcontroller is deemed sufficient as user input. The UART protocol is chosen due to previous experience with the protocol, making it an obvious choice. 
%PID-controller
%single, cascade
%Current
%Test a lot of different tuning methods



\subsection*{Embedded System}
It is chosen to use FreeRTOS v10.2.1 as operating system for the microcontroller, because it offers key features for implementing multiple tasks to run in parallel. Furthermore FreeRTOS introduces prioritisation of task without compromising the timing of individual tasks when having multiple time critical tasks. Beyond that FreeRTOS provides functionality for queues, mutex' and semaphores. When using the preemptive FreeRTOS, it must be ensured that the \textit{overhead} is not an issue.

% Particularly; preemptive scheduling supporting multitasking, prioritisation of tasks and by being a real-time OS .

% Beyond that FreeRTOS provides functionality for queues, mutex' and semaphores. It is important to notice however that CPU utilisation cannot be too high since it could result in starvation.
In the project description section \ref{sec:ProblemDescription} it is stated that a SPI connection has to be used for exchange of data between the FPGA and the microcontroller. Therefore hardware \textit{modules} supporting the SPI protocol has to be implemented on the FPGA. Furthermore the FPGA must be able to interface the motor encoder keeping track of the position and the velocity. The FPGA also has to produce the PWM-signal controlling the motor, the frequency of the signal has to comply with the specification of the H-bridge introduced in section \ref{sec:Analysis}. For regulating the pan-tilt system to a desired position it is essential to have a reference position. To be able to obtain the reference position, reset functionality is to be build into the FPGA, to ensure that the pan-tilt system will return to a defined zero position. 


%Therefore it is necessary to implement modules capable of constituting such a protocol. Beside handling the communication the FPGA must be able to deduct the position and velocity using the encoders signals and producing a desired PWM signal to the H-brigde. In accordance with the datasheet of the H-Bridge, L6203 \ref{sec:digital_appendix}, it requires a PWM frequency typical around 30 KHz and up to 100 KHz to function, meaning that the PWM signal from the FPGA must be in that frequency range. 

%It is chosen to divide the program implemented in the FPGA in to small modules that only contains the necessary code specific for that module. The intent of this is to make most of the modules reusable controlling the pan frame as well as the tilt frame.

%Krav omkring reset til index pos.
% For regulating the pan-tilt system to a desired position it is essential to have a reference position. To be able to obtain the reference position, a reset functionality is to be build into the FPGA, to ensure that the pan-tilt system will return to a defined zero position. 

%Køre SPI 
%Sende PWM (opløsning) med bestemt frekvens
%Beregne position og hastighed
%Program opbygget i moduler (nemt er organisere og implementere på pan og tilt)

\end{document}