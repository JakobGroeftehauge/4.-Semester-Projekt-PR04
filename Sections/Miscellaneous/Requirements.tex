\documentclass[../../main.tex]{subfiles}

\begin{document}
In accordance with section \ref{sec:ProblemDescription} and section \ref{sec:Delimination}, different PID-controllers are to be designed and the following must be examined:
\begin{itemize}
    \item The effect of different tuning methods for the controller, namely the Ziegler-Nichols and pole placement method.
    \item The effect of using a single controller design versus a cascaded controller design with a velocity controller in the inner loop and a position controller in the outer loop.
\end{itemize}

The following specifications shall be used as guidelines
when designing the PID-controllers:
\begin{itemize}
    %\item The designed PID-controllers are to be stable.
    \item The controllers are to have a maximum rise time of $\SI{0.5}{\second}$, a maximum overshoot of \SI{5}{\percent} and a maximum \SI{1}{\percent} settling time of $\SI{1.5}{\second}$.
\end{itemize}

To meet the aforementioned goals, a program which allows for processing PID-controllers must be implemented on the microcontroller. To ensure reliable performance, the following requirements must be met:

\begin{itemize}
    \item The PID-controllers must be implemented on the microcontroller.
    \begin{itemize}
        \item The PID-controllers must not utilize more than \SI{60}{\percent} of the CPU time on the microcontroller.
    \end{itemize}
    \item The microcontroller must run the FreeRTOS operating system.
    %\item The program running on the microcontroller must consist of loosely coupled tasks scheduled by the FreeRTOS operating system.
    \begin{itemize}
        \item It must be verified that the overhead does not compromise the timing of the system.
    \end{itemize}
   % \item The PID-controllers must not utilize more than \SI{60}{\percent} of the CPU time on the microcontroller. 
    %\item The microcontroller is not to utilise more than \SI{80}{\percent} of the total CPU time.
    \item All PID-controllers must be processed within a specified time frame to guarantee a minimum frequency. The minimum frequency and time period is to be determined.
    
    \item A UART protocol is to be implemented to allow for user input.
\end{itemize}
Using an FPGA the motor encoders must be interfaced to provide feedback for the PID-controllers executing on the microcontroller. The interface must meet the following requirements:
\begin{itemize}
    \item Data transmission between the FPGA and the microcontroller must be possible through SPI.
    \item Using SPI with a transmission frequency of \SI{1}{\kilo\hertz}, the FPGA is to receive 10000/10000 messages correct with no errors.
    \item The FPGA is to be able to keep track of the position and velocity of the motors via the encoders.
    \item Provided with a duty cycle from the microcontroller, the FPGA must be able to generate a PWM signal with a frequency between \SI{30}{\kilo\hertz} and \SI{100}{\kilo\hertz}.
    \item A reset function for the pan-tilt system is to be implemented to allow for a reference position.
\end{itemize}

\end{document}