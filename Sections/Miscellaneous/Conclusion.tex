\documentclass[../../main.tex]{subfiles}
\begin{document}
% Maybe consider a diffrerent strucure.
    % Start out by concluding on the general framework that has been built to enable the controller to be built. Here the FPGA must be mentioned as the interface/link. Choices which have been made on the microcontroller also. How do the two communicate?
    % Lead into the controller design. were the specifications met? Which controller design performed the best?
    % conclude what caused some issues.

%general overview
Using the ARTIX-7 FPGA and the TM4C123GH6P microcontroller an interface to the motor system, providing the required logic to facilitate PID-controllers, has been designed. 

A program containing tasks for handling data transmission via SPI, communication with a user through UART and execution of the PID-controller algorithms has been developed and implemented on the microcontroller. The tasks are designed to be loosely coupled, and are scheduled by the freeRTOS operating system.

An average CPU utilisation of \SI{54}{\percent} is determined satisfying the requirement of a maximum CPU utilisation of \SI{80}{\percent}. An approximate overhead of \SI{5,7}{\percent} is found and considering the CPU utilisation, it is deemed, that the overhead does not compromise the timing of the system. A UI task, utilising a UART driver handling the peripheral UART modules, has been developed to manage different user commands, for which a protocol has been designed.

Functionality has been developed to allow for communication between the microcontroller and the FPGA via the SPI protocol. The SPI communication is shown to transmit and receive 10000/10000 consistently between the microcontroller and the FPGA at a transmission frequency of \SI{2}{\kilo\hertz}, thus complying with the requirement.

Functionality has been implemented on the FPGA to interface the motor encoders and keep track of the motor positions by counting the number of encoder ticks relative to some reference position. Furthermore functionality to derive the velocity of the motors, based on the time between encoder ticks, has been implemented. This method of obtaining the velocity has been found to be erroneous, and the effects of this are discussed in the report. 

To control the motors functionality is designed producing a phase correct PWM signal with a frequency of \SI{40}{\kilo\hertz}. The PWM signal is dictated by a duty cycle received through SPI. To be able to deduct an initial reference position, a reset function is implemented on the FPGA.

Mathematical models of the pan and tilt motors have been developed to allow for the systems to be simulated, and to be able to design PID-controllers using a pole placement method.
Single position PID-controllers have been designed for both the pan and tilt motors based on the mathematical model. Furthermore, a velocity PID-controller have been designed for the tilt motor based on the mathematical model as well as the heuristic Ziegler-Nichols method. Lastly, using each of the two velocity controllers as the inner controller in a cascaded controller design, outer position controllers have been designed based on the mathematical model. Already in the design phase it proved somewhat difficult to thoroughly meet all the specifications, while still maintaining low enough gains that the required control signals were reasonable.
The designed controllers have been implemented into the microcontroller and have been tested to evaluate the performance and compare it to the performance predicted by the model.
Here it was found that the tilt motor model seemed to be quite accurate, while the model of the pan motor seemed a bit more off. %indsæt facts

Contrary to expectations, it was found that the cascaded controller design did not improve performance over the single controller design, while at the same time being more complex to design and tune. Tests also showed that good performance of the inner controllers are crucial to a cascaded design. The Ziegler-Nichols design method gave significantly better results for designing the velocity controller compared to the model based design. This may partly be caused by an erroneous velocity feedback, the effects of which have been discussed in the report.

%overshoot - 0.8. settling time - 2.86. rise time -  0.34

%tuning - results
%The Ziegler-Nichols and pole placement tuning methods are used to visualise the benefits and examine the effects on the response of the system. All simulations are made based on specifications of \SI{5}{\percent} overshoot, \SI{1,5}{\second} settling time and \SI{0,5}{\second} rise time. From the tests made in the project it is apparent that it is imperative to have knowledge of the plant to achieve a true simulation of the system. None of the tests using pole placement or Ziegler-Nichols yields a response, satisfies the requirement specifications. However it is shown that the single position controller using pole placement on the tilt motor has the best result with an overshoot of \SI{0,8}{\percent}, a settling time of \SI{2,86}{\second} and a rise time of \SI{0,34}{\second}. Being the system with most known parameters, it complies well with aforementioned statement.   %model vs. implementation

%implementation - 
%The PID-controllers are designed to control the position and velocity of the pan-tilt system and implemented using the Tustin's method. It is examined whether a single PID-controller is better than two PID-controllers in cascade. According to the tests a cascaded controller will not improve performance of the system in comparison with a single controller. The best performing cascade controller achieved an overshoot of \SI{14}{\percent}, a settling time of \SI{3,22}{\second} and a rise time of \SI{0,38}{\second}. However this does not comply with the requirement specifications. Ideally the cascaded PID-controller should improve the response of the controller, however due to imperfections in the encoders, it has not been able to achieve a satisfying response for velocity controller.  %conclusion here!

%In conclusion different methods of control theory have been examined. In spite of modelling the controllers to satisfy the requirement specifications, the implemented controllers do not comply. However it has been possible to achieve an implementation of a PID-controller, which resembles the response of simulated controllers. The cause is discussed further in the report and possible solutions are presented. Based on the tests presented in this report it seems the added velocity controller of a cascaded controller design does not yield a better overall performance, while making tuning more complex.

% FPGA - possible to control the motors (PWM), SPI,encoders, reset.
% microcontroller - Scheduling, Overhead, SPI, cpu time/utilisation.
% PID-controller - different methods. - which controllers are made?
%   - Ziegler-Nichols: results model vs. implementation
%   - Pole placement: results model vs. implementation
%   - Cascade: results? better than single controller?
%   - Did we meet the requirements in relation to overshoot, settling time and rise time?
% Other remarks

\end{document}